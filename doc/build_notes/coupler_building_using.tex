%        File: building_using.tex
%
\documentclass[letterpaper]{article}
\usepackage[top=1.0in,bottom=1.0in,left=1.0in,right=1.0in]{geometry}
\usepackage{verbatim}
\usepackage{amssymb}
\usepackage{graphicx}
\usepackage{longtable}
\usepackage{amsfonts}
\usepackage{amsmath}
\usepackage[usenames]{color}
\usepackage[
naturalnames = true, 
colorlinks = true, 
linkcolor = black,
anchorcolor = black,
citecolor = black,
menucolor = black,
urlcolor = blue
]{hyperref}
\usepackage{listings}
\usepackage{textcomp}
\definecolor{listinggray}{gray}{0.9}
\definecolor{lbcolor}{rgb}{0.9,0.9,0.9}
\lstset{
  backgroundcolor=\color{lbcolor},
  tabsize=4,
  rulecolor=,
  language=c++,
  basicstyle=\scriptsize,
  upquote=true,
  aboveskip={1.5\baselineskip},
  columns=fixed,
  showstringspaces=false,
  extendedchars=true,
  breaklines=true,
  prebreak =
  \raisebox{0ex}[0ex][0ex]{\ensuremath{\hookleftarrow}},
  frame=single,
  showtabs=false,
  showspaces=false,
  showstringspaces=false,
  identifierstyle=\ttfamily,
  keywordstyle=\color[rgb]{0,0,1},
  commentstyle=\color[rgb]{0.133,0.545,0.133},
  stringstyle=\color[rgb]{0.627,0.126,0.941},
}

\author{Stuart R. Slattery
\\ \href{mailto:sslattery@wisc.edu}{\texttt{sslattery@wisc.edu}}
}

\date{Wednesday December 14, 2011}
\title{Using the Coupling Package}
\begin{document}
\maketitle

\section{Introduction}
As outlined in the companion document {\sl Design and Implementation of
the Coupling Pacakge}, the coupling package is intended for use as a
transfer mechanism in multiphysics simulations. This document
describes the software infrastructure used in the development of the
coupling package, instructions on obtaining and building the coupling
package, and an outline of how to use the package presented through a
concrete example. This example has been demonstrated on a dual core laptop,
quad core workstation, and 100 cores on a small cluster using varying
MPI implementations.

\section{Obtaining the Coupling Package}
The coupling package is version controlled with a git software
repository residing on the casl-dev server. Changes to this repository
are communicated to a CASL hosted and moderated mailing list. Access
requests to this list can be made at
http://casl-dev.ornl.gov/mailman/listinfo/coupler-infrastructure.

Due to its direct dependencies on the Trilinos packages Teuchos and
Tpetra, the first step in obtaining the coupling package is to obtain
Trilinos. Recent changes to the Trilinos repository to allow the
coupling package to behave as a third party library (TPL) require a
recent repository clone of Trilinos. Beginning in a working directory
of your choice, the following steps are required.

\paragraph{git clone /casl-dev/git-root/Trilinos Trilinos}
Clone a copy of the Trilinos git repository.

\paragraph{cd Trilinos}
Move into the Trilinos directory.

\paragraph{git clone /casl-dev/git-root/coupler Coupler} 
Clone a copy of the coupler repository into the Trilinos home
directory.

\section{Building the Coupling Package}
Once the Trilinos and Coupler repositories have been cloned, both are
ready to be configured. The coupling package uses the Tribits build
system based on cmake and configures as a Trilinos TPL.

\subsection{Configure}
Consider an out of source build in a /coupler\_build directory. The
following shell script, also found in
/Coupler/doc/build\_notes/sample\_cmake\_configure.sh, configures
Trilinos and the Coupler for a parallel debug build using an MPI
implementation. To date, the coupler has been tested with both OpenMPI
and MPICH2 implementations on shared and distributed memory hardware.

\begin{verbatim}
#!/bin/bash

cmake \
-D CMAKE_INSTALL_PREFIX:PATH=/coupler_build \
-D CMAKE_BUILD_TYPE:STRING=DEBUG \
-D CMAKE_VERBOSE_MAKEFILE:BOOL=ON \
-D Trilinos_ENABLE_TESTS:BOOL=OFF \
-D Trilinos_ENABLE_Teuchos:BOOL=ON \
-D Teuchos_ENABLE_EXTENDED:BOOL=ON \
-D Trilinos_ENABLE_Tpetra:BOOL=ON \
-D Trilinos_ENABLE_ALL_OPTIONAL_PACKAGES:BOOL=OFF \
-D TPL_ENABLE_MPI:BOOL=ON \
-D MPI_BASE_DIR:PATH=MPI_INSTALL_DIR \
-D BLAS_LIBRARY_DIRS:PATH=BLAS_INSTALL_DIR \
-D BLAS_LIBRARY_NAMES:STRING="blas" \
-D LAPACK_LIBRARY_DIRS:PATH=LAPACK_INSTALL_DIR \
-D LAPACK_LIBRARY_NAMES:STRING="lapack" \
-D Trilinos_EXTRA_REPOSITORIES="Coupler" \
-D Trilinos_ENABLE_coupler:BOOL=ON \
-D coupler_ENABLE_TESTS:BOOL=ON \
/Trilinos
\end{verbatim}

Aside from a standard Trilinos configuration to build Teuchos and
Tpetra, the only additions are the last three lines enabling the
Coupler as an extra repository and enable the coupler unit tests.

\subsection{Build}

After configuration, {\sl make} will build the library while {\sl make
  install} will install it.

\subsection{Test}
Unit tests in both serial and parallel have been implemented using the
Teuchos unit testing harness and are incorporated into the Tribits
build. Using the configuration above, after building the tests can be
invoked with ctest. If successful, you should see an output like that
presented below.

\begin{verbatim}
stuart@dionysus:/local.hd/cnergg/stuart/builds/coupler$ ctest
Test project /local.hd/cnergg/stuart/builds/coupler
    Start 1: couplerMesh_Point_test_MPI_4
1/5 Test #1: couplerMesh_Point_test_MPI_4 ...............   Passed    1.05 sec
    Start 2: couplerMesh_Bounding_Box_test_MPI_4
2/5 Test #2: couplerMesh_Bounding_Box_test_MPI_4 ........   Passed    1.04 sec
    Start 3: couplerCore_Interfaces_test_MPI_4
3/5 Test #3: couplerCore_Interfaces_test_MPI_4 ..........   Passed    1.05 sec
    Start 4: couplerCore_Data_Field_test_MPI_4
4/5 Test #4: couplerCore_Data_Field_test_MPI_4 ..........   Passed    1.05 sec
    Start 5: couplerCore_Advanced_Transfer_test_MPI_4
5/5 Test #5: couplerCore_Advanced_Transfer_test_MPI_4 ...   Passed    1.05 sec

100% tests passed, 0 tests failed out of 5

Label Time Summary:
coupler    =   5.24 sec

Total Test time (real) =   5.25 sec
\end{verbatim}

\section{Using the Coupling Package}

\section{Conclusion}

\pagebreak
\bibliographystyle{ieeetr}
\bibliography{paper}
\end{document}


