%        File: building_using.tex
%
\documentclass[letterpaper]{article}
\usepackage[top=1.0in,bottom=1.0in,left=1.0in,right=1.0in]{geometry}
\usepackage{verbatim}
\usepackage{amssymb}
\usepackage{graphicx}
\usepackage{longtable}
\usepackage{amsfonts}
\usepackage{amsmath}
\usepackage[usenames]{color}
\usepackage[
naturalnames = true, 
colorlinks = true, 
linkcolor = black,
anchorcolor = black,
citecolor = black,
menucolor = black,
urlcolor = blue
]{hyperref}
\usepackage{listings}
\usepackage{textcomp}
\definecolor{listinggray}{gray}{0.9}
\definecolor{lbcolor}{rgb}{0.9,0.9,0.9}
\lstset{
  backgroundcolor=\color{lbcolor},
  tabsize=4,
  rulecolor=,
  language=c++,
  basicstyle=\scriptsize,
  upquote=true,
  aboveskip={1.5\baselineskip},
  columns=fixed,
  showstringspaces=false,
  extendedchars=true,
  breaklines=true,
  prebreak =
  \raisebox{0ex}[0ex][0ex]{\ensuremath{\hookleftarrow}},
  frame=single,
  showtabs=false,
  showspaces=false,
  showstringspaces=false,
  identifierstyle=\ttfamily,
  keywordstyle=\color[rgb]{0,0,1},
  commentstyle=\color[rgb]{0.133,0.545,0.133},
  stringstyle=\color[rgb]{0.627,0.126,0.941},
}

%%---------------------------------------------------------------------------%%
\author{Stuart R. Slattery
  \\ \href{mailto:sslattery@wisc.edu}{\texttt{sslattery@wisc.edu}}
}

\date{Wednesday December 20, 2011}
\title{Using the Coupling Package}
\begin{document}
\maketitle

%%---------------------------------------------------------------------------%%
\section{Introduction}
As outlined in the companion document {\sl Design and Implementation of
  the Coupling Pacakge}, the coupling package is intended for use as a
transfer mechanism in multiphysics simulations. This document
describes the software infrastructure used in the development of the
coupling package, instructions on obtaining, building, and testing the
coupling package, and an outline of how to use the package presented
through a concrete example. This example has been demonstrated to
operate in parallel on various hardware elements from the desktop
level to a small scale development cluster, including the CASL fissile
four development machines. In addition, this document should serve as
an initial report on progress towards applying the VERA software quality
assurance practices.

%%---------------------------------------------------------------------------%%
\section{Obtaining the Coupling Package}
The coupling package is version controlled with a git software
repository residing on the casl-dev server. Changes to this repository
are communicated to a CASL hosted and moderated mailing list. Access
requests to this list can be made at
{\sl http://casl-dev.ornl.gov/mailman/listinfo/coupler-infrastructure}. In 
addition, progress is begin made toward including the coupling package
in the VERA continuous integration and nightly testing environment and
completion of this task is expected soon.

To begin, the VERA repository is required as the base infrastructure
for building and executing the coupling package as a VERA
component. Due to its direct dependencies on the Trilinos packages
Teuchos and Tpetra, the second step in obtaining the coupling package
is to obtain Trilinos. Using the fissile four development machines and
the associated development environment, the following steps are
required to build VERA with the coupling package. For purposes of
testing this documentation, the machine u233 was used.

\paragraph{git clone /casl-dev/git-root/VERA VERA}
Clone a copy of the VERA git repository.

\paragraph{cd VERA}
Move into the VERA directory.

\paragraph{git clone /casl-dev/git-root/Trilinos Trilinos}
Clone a copy of the Trilinos git repository into the VERA home
directory. 

\paragraph{git clone /casl-dev/git-root/coupler Coupler} 
Clone a copy of the coupler repository into the VERA home
directory.

%%---------------------------------------------------------------------------%%
\section{Building the Coupling Package}
Once the Trilinos and Coupler repositories have been cloned, both are
ready to be configured with VERA. The coupling package uses the
Tribits build system based on cmake and configures as a VERA TPL.

%%---------------------------------------------------------------------------%%
\subsection{Configure}
Consider an out of source build in a {\sl /vera\_build} directory. The
following shell script, also found in the location
{\sl /Coupler/doc/build\_notes/sample\_cmake\_configure.sh}, configures
Trilinos and the Coupler for a parallel debug build using the default MPI
implementation in the fissile four development environment. To date,
the coupling package has been tested with both OpenMPI and MPICH2
implementations of the MPI specification on shared and distributed
memory hardware.  

\begin{verbatim}
#!/bin/bash

cmake \
-D CMAKE_INSTALL_PREFIX:PATH=/vera_build          \
-D CMAKE_BUILD_TYPE:STRING=DEBUG                  \
-D CMAKE_VERBOSE_MAKEFILE:BOOL=ON                 \
-D TPL_ENABLE_MPI:BOOL=ON                         \
-D VERA_EXTRA_REPOSITORIES="Trilinos;Coupler"     \
-D Teuchos_ENABLE_EXTENDED:BOOL=ON                \
-D VERA_ENABLE_coupler:BOOL=ON                    \
-D coupler_ENABLE_TESTS:BOOL=ON                   \
-D coupler_ENABLE_EXAMPLES:BOOL=ON                \
/VERA
\end{verbatim}

Aside from a standard VERA configuration with Trilinos, we designate
that the coupler is an extra VERA repository, the Teuchos package in
Trilinos should be extended to enable the communication utilities, and
the coupling package should be enabled by VERA with tests and examples
on.

%%---------------------------------------------------------------------------%%
\subsection{Build}

After configuration, {\sl make} will build the library while {\sl make
  install} will install it in the specified prefix location.

%%---------------------------------------------------------------------------%%
\subsection{Test}
Unit tests in both serial and parallel have been implemented using the
Teuchos unit testing harness and are incorporated into the Tribits
build. Using the configuration above, after building the tests can be
invoked with ctest. If successful, you should see an output like that
presented below. Note below that the WaveDamper example to be later
discussed was executed as well.

\begin{verbatim}
stuart@beaker:~/software/builds/VERA$ ctest
Test project /home/stuart/software/builds/VERA
    Start 1: couplerMesh_Point_test_MPI_4
1/6 Test #1: couplerMesh_Point_test_MPI_4 ...............   Passed    1.26 sec
    Start 2: couplerMesh_Bounding_Box_test_MPI_4
2/6 Test #2: couplerMesh_Bounding_Box_test_MPI_4 ........   Passed    1.24 sec
    Start 3: couplerCore_Interfaces_test_MPI_4
3/6 Test #3: couplerCore_Interfaces_test_MPI_4 ..........   Passed    1.22 sec
    Start 4: couplerCore_Data_Field_test_MPI_4
4/6 Test #4: couplerCore_Data_Field_test_MPI_4 ..........   Passed    1.42 sec
    Start 5: couplerCore_Advanced_Transfer_test_MPI_4
5/6 Test #5: couplerCore_Advanced_Transfer_test_MPI_4 ...   Passed    1.36 sec
    Start 6: couplerCore_WaveDamperExample_MPI_4
6/6 Test #6: couplerCore_WaveDamperExample_MPI_4 ........   Passed    1.43 sec

100% tests passed, 0 tests failed out of 6

Label Time Summary:
coupler    =   7.95 sec

Total Test time (real) =  10.97 sec
\end{verbatim}

%%---------------------------------------------------------------------------%%
\section{Using the Coupling Package}
To demonstrate using the coupling package in a multiphysics
simulation, a simple example was constructed to show both
implementations of the {\sl Data\_Source} and {\sl Data\_Target}
interfaces specified by the companion document {\sl Design and
  Implementation of the Coupling Pacakge} and a multiphysics driver
utilizing the coupling package to perform data transfer in an
iterative manner.

%%---------------------------------------------------------------------------%%
\subsection{WaveDamper Example}
Provided as an example in the coupler repository and enabled in the
above configuration, the WaveDamper example exists in the directory
{\sl /Coupler/example/WaveDamper}. Two simple codes are coupled in the
simulation; a code called Wave that computes a one dimensional cosine
shape over the parallel domain and another code called Damper that
computes the amount by which the amplitude of an incoming function
should be decreased. They are coupled through a Picard iteration in
the following fashion until the amplitude of the wave is dampened
below a specified tolerance.

\begin{verbatim}
Wave computes its initial conditions.

while the amplitude of the wave computed by Wave is > tolerance:

    Wave transfers its solution to Damper.

    Damper computes the amount by which the amplitude of the Wave
    solution should be decreased.

    Damper transfers the computed dampening back to Wave.

    Wave applies the dampening to compute a new solution.
\end{verbatim}

Here, the convergence tolerance on the wave amplitude is specified as
1.0E-6 in the WaveDamper implementation.

%%---------------------------------------------------------------------------%%
\subsubsection{Wave Implementation}
The Wave was implemented as shown in the following listings. Note the
methods included to provide access to various members relating to data
and the grid that have been added. These are similar to the types of
methods that will be required during the refactoring process necessary
for using the coupling package. It should also be noted that the
coupling package does not place any requirements on solve methods. For
the Wave code, the L2 norm of the residual returned for determining
convergence, not data transfer.

\begin{lstlisting}
  #include <vector>

  #include "Teuchos_RCP.hpp"
  #include "Teuchos_Comm.hpp"

  class Wave
  {
    private:
    
    Teuchos::RCP<const Teuchos::Comm<int> > comm;
    std::vector<double> grid;
    std::vector<double> f;
    std::vector<double> damping;

    public:

    Wave(Teuchos::RCP<const Teuchos::Comm<int> > _comm,
    double x_min,
    double x_max,
    int num_x);

    ~Wave();

    // Get the communicator.
    Teuchos::RCP<const Teuchos::Comm<int> > get_comm()
    {
      return comm;
    }

    // Get a const reference to the local grid.
    const std::vector<double>& get_grid()
    {
      return grid;
    }

    // Get a reference to the local data.
    std::vector<double>& get_f()
    {
      return f;
    }

    // Apply the damping to the local data structures from an external
    // source. 
    std::vector<double>& set_damping()
    {
      return damping;
    }

    // Solve the local problem and return the l2 norm of the local residual.
    double solve();
  };
\end{lstlisting}

\begin{lstlisting}
  #include "Wave.hpp"

  #include <algorithm>
  #include <cmath>

  Wave::Wave(Teuchos::RCP<const Teuchos::Comm<int> > _comm,
  double x_min,
  double x_max,
  int num_x)
  : comm(_comm)
  {
    // Create the grid.
    grid.resize(num_x);
    double x_size = (x_max - x_min) / (num_x);

    std::vector<double>::iterator grid_iterator;
    int i = 0;

    for (grid_iterator = grid.begin();
    grid_iterator != grid.end();
    ++grid_iterator, ++i)
    {
      *grid_iterator = i*x_size + x_min;
    }

    // Set initial conditions.
    damping.resize(num_x);
    std::fill(damping.begin(), damping.end(), 0.0);
    f.resize(num_x);
    std::vector<double>::iterator f_iterator;
    for (f_iterator = f.begin(), grid_iterator = grid.begin();
    f_iterator != f.end();
    ++f_iterator, ++grid_iterator)
    {
      *f_iterator = cos( *grid_iterator );
    }
  }

  Wave::~Wave()
  { /* ... */ }

  double Wave::solve()
  {
    // Apply the dampened component.
    double l2_norm_residual = 0.0;
    double f_old = 0.0;
    std::vector<double>::iterator f_iterator;
    std::vector<double>::const_iterator damping_iterator;
    for (f_iterator = f.begin(), damping_iterator = damping.begin();
    f_iterator != f.end();
    ++f_iterator, ++damping_iterator)
    {
      f_old = *f_iterator;
      *f_iterator -= *damping_iterator;
      l2_norm_residual += (*f_iterator - f_old)*(*f_iterator - f_old);
    }
    
    // Return the l2 norm of the local residual.
    return pow(l2_norm_residual, 0.5);
  }
\end{lstlisting}

%%---------------------------------------------------------------------------%%
\subsubsection{Damper Implementation}
The following listings give the implementation of the Damper
code. Again, note the methods used to provide access to the various
data structures needed by the coupling package. These methods will be
used in the interface implementations given by the following sections.

\begin{lstlisting}
  #include <vector>

  #include "Teuchos_RCP.hpp"
  #include "Teuchos_Comm.hpp"

  class Damper
  {
    private:

    Teuchos::RCP<const Teuchos::Comm<int> > comm;
    std::vector<double> wave_data;
    std::vector<double> damping;
    std::vector<double> grid;

    public:

    Damper(Teuchos::RCP<const Teuchos::Comm<int> > _comm,
    double x_min,
    double x_max,
    int num_x);

    ~Damper();

    // Get the communicator.
    Teuchos::RCP<const Teuchos::Comm<int> > get_comm()
    {
      return comm;
    }

    // Get a reference to the local damping data.
    std::vector<double>& get_damping()
    {
      return damping;
    }

    // Get a reference to the local grid.
    std::vector<double>& get_grid()
    {
      return grid;
    }

    // Get the wave data to apply damping to from an external source.
    std::vector<double>& set_wave_data()
    {
      return wave_data;
    }

    // Apply damping to the local problem.
    void solve();
  };
\end{lstlisting}

\begin{lstlisting}
  #include "Damper.hpp"

  Damper::Damper(Teuchos::RCP<const Teuchos::Comm<int> > _comm,
  double x_min,
  double x_max,
  int num_x)
  : comm(_comm)
  {
    // Create the grid.
    grid.resize(num_x);
    double x_size = (x_max - x_min) / (num_x);

    std::vector<double>::iterator grid_iterator;
    int i = 0;

    for (grid_iterator = grid.begin();
    grid_iterator != grid.end();
    ++grid_iterator, ++i)
    {
      *grid_iterator = i*x_size + x_min;
    }

    // Set initial conditions.
    damping.resize(num_x);
    wave_data.resize(num_x);
  }

  Damper::~Damper()
  { /* ... */ }

  // Apply damping to the local problem.
  void Damper::solve()
  {
    std::vector<double>::iterator damping_iterator;
    std::vector<double>::const_iterator wave_data_iterator;
    for (damping_iterator = damping.begin(),
    wave_data_iterator = wave_data.begin();
    damping_iterator != damping.end();
    ++damping_iterator, ++wave_data_iterator)
    {
      *damping_iterator = *wave_data_iterator / 2;
    }
  }
\end{lstlisting}

%%---------------------------------------------------------------------------%%
\subsubsection{Wave Interface Implementations}
As given above by the algorithm used to couple Wave to Damper, two way
transfer is required and therefore both codes must implement the {\sl
  Data\_Source} and {\sl Data\_Target} in order for this scheme to be
possible. The following listings in this section give the interface
implementations for the Wave code. For both interfaces, the scalar
transfer methods were implemented to perform no operations as this
example does not make use of them. Both implementations, though
templated to inherit from the pure virtual interface specifications,
do not use the template parameters. Rather, they explicitly specify in
the typdef block elements of type double to be transferred, int to be
used for point handles, and doubles to be used for coordinates.

To begin, the {\sl Data\_Source} implementation listing for the Wave
code has several elements that should be addressed. First, the
constructor in this implementation, and that of the {\sl Data\_Target}
implementation, takes a pointer to the wave object that it will
operate on. For the {\sl comm()} method, a simple redirection is used
to grab the communicator from the local Wave object. 

For all transfers under this interface, the name of the field being
transferred is used to identify the appropriate operations to perform
in the interface. The {\sl Wave\_Data\_Source} implementation will
serve as the source for the WAVE\_FIELD and therefore the {\sl
  field\_supported()} method designates this. In addition, all other
interface methods operating on a specific field will check for this
name before doing operations.

The {\sl get\_points()} method checks the incoming point against the
local grid. If it is found to be a local grid point, true is
returned. Finally, the {\sl send\_data()} method returns a view of the
Wave code solution, provided by a refactor method {\sl get\_f()}.

\begin{lstlisting}
  #include <algorithm>

  #include "Wave.hpp"

  #include <Mesh_Point.hpp>
  #include <Coupler_Data_Source.hpp>

  #include "Teuchos_ArrayView.hpp"
  #include "Teuchos_RCP.hpp"
  #include "Teuchos_DefaultComm.hpp"

  namespace Coupler {

    // Data_Source interface implementation for the Wave code.
    template<class DataType_T, class HandleType_T, class CoordinateType_T>
    class Wave_Data_Source 
    : public Data_Source<DataType_T, HandleType_T, CoordinateType_T>
    {
      public:

      typedef double                                   DataType;
      typedef int                                      HandleType;
      typedef double                                   CoordinateType;
      typedef int                                      OrdinalType;
      typedef Point<HandleType,CoordinateType>         PointType;
      typedef Teuchos::Comm<OrdinalType>               Communicator_t;
      typedef Teuchos::RCP<const Communicator_t>       RCP_Communicator;
      typedef Teuchos::RCP<Wave>                       RCP_Wave;

      private:

      // Wave object to operate on.
      RCP_Wave wave;

      public:

      Wave_Data_Source(RCP_Wave _wave)
      : wave(_wave)
      { /* ... */ }

      ~Wave_Data_Source()
      { /* ... */ }

      RCP_Communicator comm()
      {
	return wave->get_comm();
      }

      bool field_supported(const std::string &field_name)
      {
	bool return_val = false;

	if (field_name == "WAVE_FIELD")
	{
	  return_val = true;
	}

	return return_val;
      }

      bool get_points(const PointType &point)
      {
	bool return_val = false;

	if ( std::find(wave->get_grid().begin(), 
	wave->get_grid().end(),
	point.x() )
	!= wave->get_grid().end() )
	{
	  return_val = true;
	}

	return return_val;
      }

      const Teuchos::ArrayView<DataType> send_data(const std::string &field_name)
      {
	Teuchos::ArrayView<DataType> return_view;

	if ( field_name == "WAVE_FIELD" )
	{
	  return_view = Teuchos::ArrayView<DataType>( wave->get_f() );
	}

	return return_view;
      }

      DataType set_global_data(const std::string &field_name)
      {
	DataType return_val = 0.0;

	return return_val;
      }
    };

  } // end namespace Coupler
\end{lstlisting}

The {\sl Data\_Target} interface implementation for the Wave code
follows the same principles as the {\sl Data\_Source} implementation
given by the previous listing. In this case, the Wave code is acting
as a target for the DAMPER\_FIELD and therfore all methods operating
by field name check for this field. In addition, the {\sl set\_points}
method recasts the one dimensional grid used by the Wave code as a
vector of three dimensional point objects, storing them as member
data. Again, the refactor methods are used here to provide views into
the private data structures of Wave.

\begin{lstlisting}
  #include <algorithm>
  #include <vector>

  #include "Wave.hpp"

  #include <Mesh_Point.hpp>
  #include <Coupler_Data_Target.hpp>

  #include "Teuchos_ArrayView.hpp"
  #include "Teuchos_RCP.hpp"
  #include "Teuchos_Comm.hpp"

  namespace Coupler {

    // Data_Target interface implementation for the Wave code.
    template<class DataType_T, class HandleType_T, class CoordinateType_T>
    class Wave_Data_Target 
    : public Data_Target<DataType_T, HandleType_T, CoordinateType_T>
    {
      public:

      typedef double                                   DataType;
      typedef int                                      HandleType;
      typedef double                                   CoordinateType;
      typedef int                                      OrdinalType;
      typedef Point<HandleType,CoordinateType>         PointType;
      typedef Teuchos::Comm<OrdinalType>               Communicator_t;
      typedef Teuchos::RCP<const Communicator_t>       RCP_Communicator;
      typedef Teuchos::RCP<Wave>                       RCP_Wave;

      private:

      RCP_Wave wave;
      std::vector<PointType> local_points;

      public:

      Wave_Data_Target(RCP_Wave _wave)
      : wave(_wave)
      { /* ... */ }

      ~Wave_Data_Target()
      { /* ... */ }

      RCP_Communicator comm()
      {
	return wave->get_comm();
      }

      bool field_supported(const std::string &field_name)
      {
	bool return_val = false;

	if (field_name == "DAMPER_FIELD")
	{
	  return_val = true;
	}

	return return_val;
      }

      const Teuchos::ArrayView<PointType> 
      set_points(const std::string &field_name)
      {
	Teuchos::ArrayView<PointType> return_view;

	if ( field_name == "DAMPER_FIELD" )
	{
	  local_points.clear();
	  Teuchos::ArrayView<const DataType> local_grid( wave->get_grid() );
	  Teuchos::ArrayView<DataType>::const_iterator grid_it;
	  int n = 0;
	  int global_handle;
	  for (grid_it = local_grid.begin(); 
	  grid_it != local_grid.end();
	  ++grid_it, ++n)
	  {
	    global_handle = wave->get_comm()->getRank() *
	    local_grid.size() + n;
	    local_points.push_back( 
	    PointType( global_handle, *grid_it, 0.0, 0.0) );
	  }
	  return_view = Teuchos::ArrayView<PointType>(local_points);
	}

	return return_view;
      }

      Teuchos::ArrayView<DataType> receive_data(const std::string &field_name)
      {
	Teuchos::ArrayView<DataType> return_view;

	if ( field_name == "DAMPER_FIELD" )
	{
	  return_view = Teuchos::ArrayView<DataType>( wave->set_damping() );
	}

	return return_view;
      }

      void get_global_data(const std::string &field_name,
      const DataType &data)
      { /* ... */ }
    };

  } // end namespace Coupler
\end{lstlisting}

%%---------------------------------------------------------------------------%%
\subsubsection{Damper Interface Implementations}
This section gives the two interface implementations for the Damper
code. The same principles used in the implementations of the Wave
interfaces also apply here. Again, the {\sl Damper\_Data\_Source}
implementation will only provide data on points that exist on its
grid. If they do not, no data will be applied. As the Damper is the
source for the DAMPER\_FIELD and a target for the WAVE\_FIELD, these
string values are used accordingly within the implementations.

\begin{lstlisting}
  #include <algorithm>

  #include "Damper.hpp"

  #include <Mesh_Point.hpp>
  #include <Coupler_Data_Source.hpp>

  #include "Teuchos_ArrayView.hpp"
  #include "Teuchos_RCP.hpp"
  #include "Teuchos_DefaultComm.hpp"

  namespace Coupler {

    // Data_Source interface implementation for the Damper code.
    template<class DataType_T, class HandleType_T, class CoordinateType_T>
    class Damper_Data_Source 
    : public Data_Source<DataType_T, HandleType_T, CoordinateType_T>
    {
      public:

      typedef double                                   DataType;
      typedef int                                      HandleType;
      typedef double                                   CoordinateType;
      typedef int                                      OrdinalType;
      typedef Point<HandleType,CoordinateType>         PointType;
      typedef Teuchos::Comm<OrdinalType>               Communicator_t;
      typedef Teuchos::RCP<const Communicator_t>       RCP_Communicator;
      typedef Teuchos::RCP<Damper>                     RCP_Damper;

      private:

      // Damper object to operate on.
      RCP_Damper damper;

      public:

      Damper_Data_Source(RCP_Damper _damper)
      : damper(_damper)
      { /* ... */ }

      ~Damper_Data_Source()
      { /* ... */ }

      RCP_Communicator comm()
      {
	return damper->get_comm();
      }

      bool field_supported(const std::string &field_name)
      {
	bool return_val = false;

	if (field_name == "DAMPER_FIELD")
	{
	  return_val = true;
	}

	return return_val;
      }

      bool get_points(const PointType &point)
      {
	bool return_val = false;

	if ( std::find(damper->get_grid().begin(), 
	damper->get_grid().end(),
	point.x() )
	!= damper->get_grid().end() )
	{
	  return_val = true;
	}

	return return_val;
      }

      const Teuchos::ArrayView<DataType> send_data(const std::string &field_name)
      {
	Teuchos::ArrayView<DataType> return_view;

	if ( field_name == "DAMPER_FIELD" )
	{
	  return_view = Teuchos::ArrayView<DataType>( damper->get_damping() );
	}

	return return_view;
      }

      DataType set_global_data(const std::string &field_name)
      {
	DataType return_val = 0.0;

	return return_val;
      }
    };

  } // end namespace Coupler
\end{lstlisting}

As in the {\sl Wave\_Data\_Target} implementation, the Damper
implementation utilizes the {\sl Mesh\_Point} container to provide the
expected representation of its grid to the interfaces.

\begin{lstlisting}
  #include <algorithm>
  #include <vector>

  #include "Damper.hpp"

  #include <Mesh_Point.hpp>
  #include <Coupler_Data_Target.hpp>

  #include "Teuchos_ArrayView.hpp"
  #include "Teuchos_RCP.hpp"
  #include "Teuchos_DefaultComm.hpp"

  namespace Coupler {

    // Data_Target interface implementation for the Damper code.
    template<class DataType_T, class HandleType_T, class CoordinateType_T>
    class Damper_Data_Target 
    : public Data_Target<DataType_T, HandleType_T, CoordinateType_T>
    {
      public:

      typedef double                                   DataType;
      typedef int                                      HandleType;
      typedef double                                   CoordinateType;
      typedef int                                      OrdinalType;
      typedef Point<HandleType,CoordinateType>         PointType;
      typedef Teuchos::Comm<OrdinalType>               Communicator_t;
      typedef Teuchos::RCP<const Communicator_t>       RCP_Communicator;
      typedef Teuchos::RCP<Damper>                       RCP_Damper;

      private:

      RCP_Damper damper;
      std::vector<PointType> local_points;

      public:

      Damper_Data_Target(RCP_Damper _damper)
      : damper(_damper)
      { /* ... */ }

      ~Damper_Data_Target()
      { /* ... */ }

      RCP_Communicator comm()
      {
	return damper->get_comm();
      }

      bool field_supported(const std::string &field_name)
      {
	bool return_val = false;

	if (field_name == "WAVE_FIELD")
	{
	  return_val = true;
	}

	return return_val;
      }

      const Teuchos::ArrayView<PointType> 
      set_points(const std::string &field_name)
      {
	Teuchos::ArrayView<PointType> return_view;

	if ( field_name == "WAVE_FIELD" )
	{
	  local_points.clear();
	  Teuchos::ArrayView<const DataType> local_grid( damper->get_grid() );
	  Teuchos::ArrayView<DataType>::const_iterator grid_it;
	  int n = 0;
	  int global_handle;
	  for (grid_it = local_grid.begin(); 
	  grid_it != local_grid.end();
	  ++grid_it, ++n)
	  {
	    global_handle = damper->get_comm()->getRank() *
	    local_grid.size() + n;
	    local_points.push_back( 
	    PointType( global_handle, *grid_it, 0.0, 0.0) );
	  }
	  return_view = Teuchos::ArrayView<PointType>(local_points);
	}

	return return_view;
      }

      Teuchos::ArrayView<DataType> receive_data(const std::string &field_name)
      {
	Teuchos::ArrayView<DataType> return_view;

	if ( field_name == "WAVE_FIELD" )
	{
	  return_view = Teuchos::ArrayView<DataType>( damper->set_wave_data() );
	}

	return return_view;
      }

      void get_global_data(const std::string &field_name,
      const DataType &data)
      { /* ... */ }
    };

  } // end namespace Coupler
\end{lstlisting}

%%---------------------------------------------------------------------------%%
\subsubsection{Coupled Driver}
Once the interfaces have been implemented for both codes, the
infrastructure is in place to do a simple coupled problem. The
following listing gives the multiphysics driver used in the WaveDamper
problem.

After parallel initialization, a domain from 0.0 to 5.0 is distributed
in parallel across each process for each code. The bounds of the
computed local domain are applied in the constructor for each
code. Following this, the WAVE\_FIELD is setup for transfer. An
instance of the {\sl Data\_Source} interface is created from the {\sl
  Wave\_Data\_Source} implementation and {\sl Data\_Target} interface
is created from the {\sl Damper\_Data\_Target} implementation. A {\sl
  Data\_Field} is then constructed to transfer the WAVE\_FIELD from
the Wave code to the Damper code. The same process is repeated for the
DAMPER\_FIELD. 

Once the various coupling package components have been initialized, an
iteration loop is setup to run until either the L2 norm of the wave
amplitude computed by the Wave code is less than 1.0E-6 or a specified
maximum number of iterations has been reached, each time executing the
algorithm specified by the picard iteration pseudocode above.

\begin{lstlisting}
  #include <iostream>

  #include "Wave.hpp"
  #include "Wave_Source.hpp"
  #include "Wave_Target.hpp"
  #include "Damper.hpp"
  #include "Damper_Source.hpp"
  #include "Damper_Target.hpp"

  #include <Coupler_Data_Source.hpp>
  #include <Coupler_Data_Target.hpp>
  #include <Coupler_Data_Field.hpp>

  #include "Teuchos_RCP.hpp"
  #include "Teuchos_CommHelpers.hpp"
  #include "Teuchos_DefaultComm.hpp"
  #include "Teuchos_GlobalMPISession.hpp"

  //---------------------------------------------------------------------------//
  // Main function driver for the coupled Wave/Damper problem.
  int main(int argc, char* argv[])
  {
    // Setup communication.
    Teuchos::GlobalMPISession mpiSession(&argc,&argv);
    Teuchos::RCP<const Teuchos::Comm<int> > comm = 
    Teuchos::DefaultComm<int>::getComm();

    // Set up the parallel domain.
    double global_min = 0.0;
    double global_max = 5.0;
    int myRank = comm->getRank();
    int mySize = comm->getSize();
    double local_size = (global_max - global_min) / mySize;
    double myMin = myRank*local_size + global_min;
    double myMax = (myRank+1)*local_size + global_min;

    // Setup a Wave.
    Teuchos::RCP<Wave> wave =
    Teuchos::rcp( new Wave(comm, myMin, myMax, 10) );

    // Setup a Damper.
    Teuchos::RCP<Damper> damper =
    Teuchos::rcp( new Damper(comm, myMin, myMax, 10) ); 

    // Setup a Wave Data Source for the wave field.
    Teuchos::RCP<Coupler::Data_Source<double,int,double> > wave_source = 
    Teuchos::rcp( new Coupler::Wave_Data_Source<double,int,double>(wave) );

    // Setup a Damper Data Target for the wave field.
    Teuchos::RCP<Coupler::Data_Target<double,int,double> > damper_target = 
    Teuchos::rcp( new Coupler::Damper_Data_Target<double,int,double>(damper) );

    // Setup a Data Field for the wave field.
    Coupler::Data_Field<double,int,double> wave_field(comm,
    "WAVE_FIELD",
    wave_source,
    damper_target);

    // Setup a Damper Data Source for the damper field.
    Teuchos::RCP<Coupler::Data_Source<double,int,double> > damper_source = 
    Teuchos::rcp( new Coupler::Damper_Data_Source<double,int,double>(damper) );

    // Setup a Wave Data Target for the damper field.
    Teuchos::RCP<Coupler::Data_Target<double, int, double> > wave_target = 
    Teuchos::rcp( new Coupler::Wave_Data_Target<double,int,double>(wave) );

    // Setup a Data Field for the damper field.
    Coupler::Data_Field<double,int,double> damper_field(comm,
    "DAMPER_FIELD",
    damper_source,
    wave_target);

    // Iterate between the damper and wave until convergence.
    double local_norm = 0.0;
    double global_norm = 1.0;
    int num_iter = 0;
    int max_iter = 100;
    while( global_norm > 1.0e-6 && num_iter < max_iter )
    {
      // Transfer the wave field.
      wave_field.transfer();

      // Damper solve.
      damper->solve();

      // Transfer the damper field.
      damper_field.transfer();

      // Wave solve.
      local_norm = wave->solve();

      // Collect the l2 norm values from the wave solve to ensure
      // convergence. 
      Teuchos::reduceAll<int>(*comm,
      Teuchos::REDUCE_MAX, 
      int(1), 
      &local_norm, 
      &global_norm);

      // Update the iteration count.
      ++num_iter;

      // Barrier before proceeding.
      Teuchos::barrier<int>( *comm );
    }

    // Output results.
    if ( myRank == 0 )
    {
      std::cout << "Iterations to converge: " << num_iter << std::endl;
      std::cout << "L2 norm:                " << global_norm << std::endl;
    }

    return 0;
  }
\end{lstlisting}

The problem was noted to converge for all cases tested in 22 iterations.

%%---------------------------------------------------------------------------%%
\subsection{WaveDamper Testing}

Using the WaveDamper example, the coupling package parallel
capabilities have been demonstrated to operate with a variety of
hardware and software components. Hardware tested includes shared
memory local workstations, the CASL fissile four machines, and a small
distributed memory cluster on the University of Wisconsin - Madison
campus. Both Intel and GNU compilers have been tested along with both
MPICH2 and OpenMPI implementations of the MPI standard. Successful
execution of the WaveDamper test was achieved on 120 cores on the
small distributed memory cluster.

%%---------------------------------------------------------------------------%%
\section{Conclusion}
A coupling package intended for use as a data transfer mechanism in
multiphysics simulations has been developed and is currently capable
of a simple form of  point based mesh coupling. The software
infrastructure used in the development of the coupling package,
instructions on obtaining, building, and testing the coupling package,
and an outline of how to use the package presented through a concrete
example have been provided as an outline for this work. The WaveDamper
example has been demonstrated to operate in parallel on various
hardware elements from the desktop level to a small scale development
cluster, including the CASL fissile four development machines. Ongoing
work will bring the coupling package under the VERA continuous
integration and nightly testing environment as well as continue to
enforce VERA software quality assurance practices. 

%%---------------------------------------------------------------------------%%
\pagebreak
\bibliographystyle{ieeetr}
\bibliography{paper}
\end{document}
